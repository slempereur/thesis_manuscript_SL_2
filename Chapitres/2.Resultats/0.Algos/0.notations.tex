\providecommand{\main}{../../..}

\documentclass[\main/main.tex]{subfiles}

\begin{document}

\begin{table}

\scriptsize

    \begin{center}
      %%% HT FAIRE DES MACROS !!
        \begin{tabular}{p{0.2\textwidth}p{0.35\textwidth}p{0.35\textwidth}}
                %Description de chaque ligne (mise en place de la tête de table)
                Notations & % Notation employee
                Noms &      % Nom de l'opération ou de l'élement décrit
                Definitions % Definition mathématique
                \\
                \midrule
                %Image
                $I$ &
                une image de $N \times M \times O$ voxels encodée en 12-bit image &
                $\forall p\in[1,N]\times[1,M]\times[1,O],\newline I(p)\in[0;4095] \cap \mathbb{Z} $ %
                \\
                %Maximum
                $ \max(I) $ &
                Valeur maximale de I &
                $ $ % Pour garder la cohérence de l'écriture
                \\
                % element wise maxima
                $ \localmax(I) $ &
                Maxima locaux de I &
                $ $
                % % element wise maxima
                % $ I_1 \bar{\bigvee} I_2 $ &
                % Element wise maximum between two images &
                % $\forall p\in I, I(p) = I_1(p)$ if $I_1(p) > I_2(p); I_2(p)$ otherwise
                \\
                %Threshold
                $(I)_{\ge \theta} $ &
                Seuillage &
                $\forall p \in I,\newline I(p)=\left\{
                    \begin{array}{ll}
                    1 & \mbox{si } I(p) \ge \theta \\
                    0 & \mbox{sinon.}
                    \end{array}
                    \right.
                \\
                %Otsu threshold
                $T_{\textnormal{Otsu}} $ &
                Seuillage d'\textsc{Otsu} &
                $ $
                \\
                %Multi class otsu threshold
                $T^n_{\textnormal{Otsu}} $ &
                Seuillage d'\textsc{Otsu} à n classes &
                $ $
                \\
                %Inversion
                $ \inv(I)$ &
                Inverse de l'image $I$ &
                $ \forall p, \inv(I(p))=4095-I(p)$
                \\
                %Pointwise maximum
                $\bigvee$ &
                Pointwise maximum &
                $ $ % Pour garder la cohérence de l'écriture
                \\
                %Pointwise menimum
                $\bigwedge$ &
                Pointwise minimum&
                $ $ % Pour garder la cohérence de l'écriture
                \\
                %cube
                $c_{i}$ &
                cube de taille $i^{3}$ &
                $ $ % Pour garder la cohérence de l'écriture
                \\
                %Dilation
                $\delta_{se}(I)$ &
                Dilatation\newline par un élement structurant se &
                $ \bigvee_{v \in se}I_{-v} $
                \\
                %Dilation
                $\delta_{I^{2}}(I)$ &
                Dilatation géodésique de $I$\newline utilisant $I^{2}$ comme masque &
                $ \bigvee_{v \in I^2}I_{-v} $
                \\
                %Erosion
                $\varepsilon_{se}(I)$ &
                Erosion\newline par un élement structurant se &
                $ \bigwedge_{v \in se}I_{v} $
                \\
                %Closing
                $\varphi_{se}(I)$ &
                Fermeture\newline par un élement structurant se &
                $\delta_{se}(\varepsilon_{se}(I))$
                \\
                %Opening
                $\gamma_{se}(I)$ &
                Ouverture\newline par un élement structurant se &
                $\varepsilon_{se}(\delta_{se}(I))$
                \\
                %moprhological gradient
                $\grad_M(I)$ &
                Gradient morphologique de $I$ &
                $\grad_M(I) = \delta_{s_1}(I) - \varepsilon_{s_1}(I)$
                \\
                % Volume opening
                $\gamma^{\alpha}_{\lambda}(I)$ &
                Ouverture par volume  &
                $ \bigvee_{se,volume(se)=\lambda}\gamma_{se}(I)$
                \\
                % UltimeVolume opening
                $\gamma^{\alpha}_{u}(I)$ &
                Ouverture ultime par volume  &
                \\% Median filter
                $ \median_{se}(I) $ &
                Filtre médian\newline utilisant un element structurant se &
                $ $ 
                \\
                % Median of array
                $ \mu = \median(I) $ &
                Médian scalaire d'une matrice I &
                $ \forall p, M(p) = \median(I)$
                \\
                $\percentile_q(I)$ &
                Qième pourcentil de l'image I &
                $ $
                \\
            \end{tabular}

        \caption{\label{tab:notations} Notations mathématiques employées.} 

    \end{center}

\end{table}

\end{document}
