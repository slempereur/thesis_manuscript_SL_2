\providecommand{\main}{../../..}
\providecommand{\Figures}{\main/Figures}

\documentclass[\main/main.tex]{subfiles}

\begin{document}
                    
\sectionmark{Adaptation de ZBI à d'autres modèles}
\section{\label{sec:modeles}ZBI pourrait être utilisée pour d'autres organismes modèles}
\sectionmark{Adaptation de ZBI à d'autres modèles}

%
L'ensemble des algorithmes présentés précédemment ont été développés sur le \pz.
%
Bien qu'il s'agisse du modèle aquatique le plus largement utilisé en laboratoire, il n'est pas nécessairement pertinent pour l'étude de la toxicologie de composés chimiques pour l'humain, du fait de sa distance évolutive, et effectuer des tests sur plusieurs espèces semble être un moyen d'éviter d'obtenir des conclusions erronées.
%
L'adaptation des travaux présentés précédemment à d'autres modèles animaux pourrait donc être réalisée.
%
Nous allons maintenant présenter ce que ZBI pourrait apporter à l'étude des trois espèces présentées en \autoref{sec:intro:modele}.

    \subsection{L'utilsation de ZBI sur le médaka ne demanderait que peu de modifications}

%%
%
Dans le cadre de ma thèse, j'ai essayé d'utiliser les algorithmes développés pour le recalage
et la segmentation de \pz{} de 5dpf sur des images de médakas à 10 dpf (voir \autoref{sec:medaka}).
%
Un exemple d'application de l'algorithme de recalage est ainsi visible en \autoref{fig:model:oz:reg}.
%
Ces données restent préliminaires, car je n'ai pas pu reproduire ces expériences sur suffisamment d'échantillons provenant d'expériences indépendantes. De plus, je n'ai pas réalisé de mesures quantitatives de la précision de segmentation, par comparaison avec des segmentations manuelles.
%
L'emploi des ces algorithmes sur cet autre organisme modèle ne demanderait donc simplement qu'une validation des algorithmes de segmentation ainsi qu'une multiplication des essais pour valider la reproductibilité des résultats obtenus.

\begin{figure}[h!]
    \centering
    \begin{subfigure}[b]{0.475\textwidth}
       \caption{
            Échantillon avant recalage
            }
       \centering \includegraphics[width=\textwidth]{\Figures/Modeles/med_dor_raw_cut.png}
    \end{subfigure}
    \begin{subfigure}[b]{0.475\textwidth}
       \caption{
            Échantillon après recalage
            }
       \centering \includegraphics[width=\textwidth]{\Figures/Modeles/med_dor_reg_cut.png}
    \end{subfigure}
    \caption{
        \label{fig:model:oz:reg}
        Application du recalage à un échantillon d'\ol{}\newline
        À la vue de ce résultat, l'utilisation des algortihmes développés pour le \pz{} semble possible sans modification.
        }
\end{figure}

    \subsection{L'adaptation de ZBI au xénope nécessiterait la création de nouveaux tampons en plus d'une modification des algorithmes}
%%
%
Ces dernières années, les méthodes de clarification de tissus mises au point chez le poisson-zèbre et la souris ont été adaptées avec succès au \xl{}~\cite{fini_2017,affaticati_2018}.
%
Il est donc déjà possible d'employer notre procédure d'\hti.
%
Il pourrait cependant être nécessaire de développer de nouveaux tampons afin de simplifier le montage des échantillons
tout en améliorant ainsi la robustesse de l'acquisition automatique.
%
Enfin, la forme du cerveau des têtards de xénopes présentant une forte différence avec le cerveau du \pz{} à 5 dpf,
une adaptation des algorithmes de segmentation risque d'être nécessaire pour permettre d'automatiser l'analyse de ces échantillons.

    \subsection{L'adaptation de ZBI à des coupes de cerveaux de souris permettrait l'étude de maladies neurodégénératives}
    
%%
De nombreuses méthodes de clarification optiques de tissus ont été mises au point chez la souris~\cite{tainaka_2016,tomer_2014}.
%
La taille des échantillons est cependant incompatible avec la mise en place de procédure d'\hti{}.
%
Il serait cependant possible d'adapter les procédures informatiques à ces échantillons.
%
Ainsi, l'utilisation du \sbddcc{} et du \sblc{} permettrait d'améliorer l'analyse de tissus transparisés.
%
Le \sblc{} pourrait permettre de réduire le temps nécessaire au marquage immuno-histochimique en compensant le gradient de marquage périphérique.
%
Le \sbddcc{} permettrait de réduire l'importance de la déperdition lumineuse pour l'imagerie d'organe entier.
%
L'adaptation des procédures de segmentations automatiques pourraient permettre d'accélérer l'analyse préliminaire des échantillons.
%
Par exemple, l'utilisation des algorithmes de segmentations sur des modèles d'étude de sclérose en plaque pourrait permettre d'effectuer des cribles de médicament permettant le traitement de cette maladie.
%
A TPS, l'imagerie de tranches de cerveaux de rats tranparisées à large échelle  a été optimisée dans le cadre d'un projet financé par la Fondation Leducq. Des méthodes de segmentation des vaisseaux ont été optimisées par Arnim Jennet afin d'étudier d'éventuels défauts des jonctions glio-vasculaires dans certains rats possédant des mutations dans la dystrophine, ne créant pas de myopathie, mais des défauts cognitifs sévères (collaboration Cyrille Vaillend).  
%
L'étude du ratio entre matière blanche et volume du cerveau pourrait ainsi à des stades relativement précoces, où le cerveau a encore une taille de seulement quelques millimètres, permettre de détecter de potentielles modifications dans le volume de myéline induites par un traitement médicamenteux.

\end{document}
