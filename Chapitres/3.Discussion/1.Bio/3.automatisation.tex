\providecommand{\main}{../..}

\documentclass[\main/main.tex]{subfiles}

\begin{document}
                    
\section{
    \label{sec:automatisation}
    Optimisation des différentes étapes du \hti
    }
 
%   
Dans l'état actuel du projet, de nombreuses étapes du protocole de \hti{} nécessitent encore de lourds et longs investissements en main d'oeuvre..
%
L'automatisation de ces étapes pourrait ainsi permettre d'augmenter le rendement de ce procédé.
%
De plus, il serait possible de libérer les opérateurs humains actuellement nécessaires,
permettant ainsi leur implication dans des projets de développement technologiques plutôt qu'aux seules routines de traitement.
%
J'envisage donc différentes voies d'amélioration permettant l'optimisation de l'\hti{}.

    \subsection[Par robots pipeteurs]{Utilisation de robot pipeteur pour le protocol \ihc{}}
    
%
Actuellement, le traitement \ihc{} des échantillons est l'étape la plus longue du protocole d'\hti{}.
%
Il représente à lui seul deux tiers du temps nécessaire au \hcs{}.
%
Malheureusement, il semble difficilement possible de réduire le temps nécessaire à cette étape,
car la plus grande partie des opérations consiste en des bains nécessairement longs,
pour que les marquages avec le marqueur lipophile ou bien les marquages anticorps puissent pénétrer dans la profondeur des tissus des larves.
%
Les temps imposés par ces marquages empêchent le lancement de ces traitements certains jours de la semaine,
car il serait alors nécessaire d'imposer à l'opérateur de travailler le week-end.

%%
De plus, avec des robots,  il serait possible de lancer un plus grand nombre de lots au même moment.
%
Actuellement, l'expérience montre qu'un opérateur humain ne peut pas traiter plus de cent larves par semaine.
%
L'imagerie d'un échantillon de \pz à cinq dpf est effectué en moyenne en huit minutes.
%
Il est donc possible d'imager cent échantillons en moins d'une journée.
%
En tenant compte du temps nécessaire au montage de ces échantillons,
à l'équilibration du milieu d'imagerie et du temps nécessaire à la mise en place des paramètres d'imagerie
il est donc possible d'imager cent échantillons par jour sur un seul microscope. 
%
De son coté, les segmentations présentées précédemment demandent dix minutes de calcul sur un ordinateur ayant une architecture standard.
%
Sans parallélisation des calculs, il est donc possible de traiter cent échantillons en environ dix-sept heures.
%
En prenant donc compte des week-ends, il serait en principe possible d'imager et de segmenter cinq cents échantillons par semaine.
%
Le traitement par \ihc{} par un opérateur humain induit donc une perte de rendement de quatre-vingts pourcents.
%
L'automatisation de ce traitement pourrait donc permettre d'atteindre les pleines capacités de traitement de ce processus sans nécessité de personnel supplémentaire.

%
Ce problème peut être résolu par l'emploi de robots pipeteurs comme ceux de la marque Eppendorf (présents à TPS) et qui sont en cours d'adaptation pour ces protocoles..
%
Ces robots,
comme l'\href{https://online-shop.eppendorf.fr/FR-fr/Automates-de-pipetage-44509.html#goto-Automates-de-pipetage-WebPMain-44509}{epmotion d'ependorf},
permettent d'automatiser les changements de milieu, d'automatiser les lavages entres les différentes étapes
et d'assurer une plus grande reproductibilité des résultats en limitant les opérations à effectuer par des opérateurs humains.
%
Ces appareils fonctionnant parfaitement seuls, il serait donc possible de lancer le traitement d'échantillons à n'importe quel moment de la semaine.
%
De plus, l'appareil effectuant seul les lavages nécessaires entre les différents bains,
l'opérateur humain serait ainsi libéré de ces étapes longues et fastidieuses.
%
De plus, la plupart de ces robots sont enfermés dans des enceintes.
%
Ce paramètre permet d'assurer un meilleur contrôle de critères comme la température,
assurant ainsi les meilleures conditions pour le traitement des échantillons.
%
Enfin, ces appareils sont pour la plupart capables de traiter plusieurs plaques en parallèles.
%
Certains modèles proposant le traitement de 5 plaques 96 puits, il est donc possible de traiter quatre cent quatre vingts échantillons en une seule fois.
%
De cette manière, il est possible d'assurer le même traitement à l'ensemble des échantillons.
%
L'utilisation d'un tel robot pourrait donc permettre de faire coïncider le nombre d'échantillons produits par semaine avec les capacités d'imagerie et de segmentations.

%
Cependant, l'emploi de tel robot demande un plus grand temps de préparation.
%
Actuellement, les échantillons sont traités en lot dans des tubes de 5 ml,
chaque tube pouvait contenir environ dix échantillons.
%
Cependant, l'utilisation de robot pipeteur nécessite souvent l'utilisation de plaque 96 puits.
%
Il est donc nécessaire de placer une larve par puits,
ce qui est aussi une tâche longue et fastidieuse.
%
Cependant, la mise en place des échantillons dans des plaques 96 puits peut être accélérée par un robot pipeteur d(embryons mis au point par le KIT à Karslruhe (citer leur site). Son acquisition avait été envisagée par TPS dans le cadre du projet d'investissement d'avenir TEFOR mais il n 'a pas été réalisé par manque de temps et de demande des utilisateurs pour des traitements à grande échelle par TPS.

    \subsection{Simplification du montage}

%    
Le montage des échantillons est actuellement réalisé par un opérateur (voir section~\ref{chapter:bio:montage}).
%
Pour rappel, il s'agit de placer manuellement chaque échantillon dans une encoche créée dans une semelle d'agarose solubilisée dans du milieu de culture d'embryons de \pzs{}.
%
Ainsi, une personne expérimentéea besoin d'une demi-journée de travail pour pouvoir monter cent échantillons.
%
De plus, la solubilisation de l'agarose dans un milieu principalement aqueux est source de problème.
%
Nous allons donc maintenant voir des solutions permettant de réduire le temps nécessaire au montage.

%% Equilibrage
La nécessité d'apparier l'indice de réfraction du milieu à l'indice de réfraction de l'échantillon ralentit le processus.
%
L'agarose étant solubilisé dans du milieu de culture d'embryons de \pz{},
la semelle utilisé comme guide possède une grande quantité d'eau.
%
Une fois recouverte de solution pour l'appariement de l'indice de réfraction (\MD{}),
l'eau présente dans la semelle diffuse dans le \MD{}, ce qui diminue l'indice de réfraction du milieu d'imagerie.
%
Avant l'imagerie,
il est donc nécessaire de réaliser plusieurs changement de milieux afin de s'assurer d'avoir l'indice de réfraction désiré.
%
Ces changements de milieu successifs induisent plusieurs problèmes. Tout d'abord, ils augmentent le temps nécessaire entre le montage et l'imagerie.
%
Ensuite, les mouvements d'eau entre la semelle et le milieu d'imagerie entraînent un stress mécanique pour les échantillons.
%
Ce stress mécanique peut alors entraîner un décollement des échantillons de la semelle d'agarose,
ou bien un décollement de la semelle du fond de la boite utilisée pour le montage.
%
Enfin, le \MD{} étant une solution très visqueuse,  l'homogénéité du milieu est difficile à assurer.
%
Il est ainsi possible que le milieu d'imagerie ne soit pas homogène, ce qui induit un indice de réfraction irrégulier dans la plaque.

%
Afin de résoudre ce problème, la solubilisation de l'agarose dans du \MD{} a été réalisé par Matthieu Simion.
%
De cette manière, la semelle d'agarose possède déjà l'indice de réfraction voulu.
%
Ceci entraîne donc une réduction du nombre de changement de bains.
Cette solution n'avait pas été envisagé précédemment car la viscosité des premières versions de MD empéchait la solubilisation de l'agarose.
%
Cependant, le développement récent du \MD{}, version du MD ayant une viscosité réduire, a permis de résoudre ce problème.

% Automatisation du montage
L'automatisation du montage est difficilement réalisable en utilisant le dispositif actuel.
%
En effet,
la forme des puits imprimés dans les semelles d'agarose empêche de monter les larves avec un volume de liquide important, d'où la stratégie de montageb avec un pinceau.
%
Si une trop grande quantité de liquide était déposé en même temps que l'échantillon,
l'échantillon risquerait de sortir de son puit ou d'être mal stabilisé par le phytagel.
%
Pour des échantillons de \pz{} à 5 dpf, les puits ont un volume inférieur à $2 mm^{3}$. 
%
Le pipetage d'échantillon de \pz{} nécessite au minimum une dizaine de microlitres, ce qui empêcherait l 'emploi de robot de pipettage de larve qui distribuent les larves dans un volume minimum d'environ 100 microlitres.
%Afin de permettre le pipetage des échantillons par un robot, il pourrait simplement être nécessaire de créer de nouveaux tampons permettant de former dans l'agarose des pyramides au dessus des puits initiales ne contenant pas d'agarose, afin de ne pas géner l'imagerie par nos très gros objectifs.
%
Aussi, en augmentant la distance existante entre deux échantillons et l'épaisseur d'agaroe,
il serait possible de pipeter une plus grande quantité de liquide tout en assurant que les échantillons ne ressortent pas des puits. Mais aussi en séchant légèrement au préalable les moules d'agarose, il est probable que l'agarose ait la capacité d'absorber le liquide déposé. Une autre possibilité, impliquant probablemlnt des imprimantes plus performantes serait d'imprimer une grille sous les puits dans un tampon qui serait placé sous le moule d'agarose. Ainsi on pourrait favoriser l'écoulement du liquide
%
%
De plus, un autre problèe du montage automatisé est l'orientation aléatoire des échantillons, certaines étant imcompatibles avec la mise en place d'une procédure de \hcs{}.
%
Ainsi, l'orientation parfaitement latérale induit par exemple des abérations lors de l'imagerie du cerveau moyen, car même si la dépigmentation a été préalablement réalisée, l'épithélium de la rétine n'est pas transparent aux rayons des lasers des microscopes confocaux et induit une perte d'information dans les tissus situés sous elle.

%
Une solution à court terme pour résoudre ce problème d'orientations pourrait être d'enlever les yeux avant imagerie, une opération relativement rapude. On pourrait aussi optimiser un protocole permettant d'imager au travers des yeux, ou finalement trouver un forme de puits adaptée permettant de contraindre la larve à être dans une orientation pseudo-dorsale dans son puits. De tels, puits ont été développés par l'équipe der Wittbrodt (ref) pour des stades plus précoces (24-48h).

%
Une autre possibilité de montage a émergé  ces dernières années.
%
Il s'agit de l'utilisation de dispositifs utilisant la microfluidique~\cite{khalili_2019}.
%
Ces dispositifs consistent en l'impression d'un piège dans un matériaux polymère, comme du polystyrène ou du polydiméthylsiloxane.
%
Dans l'utilisation la plus courante, le dispositif est recouvert d'une lamelle afin de permettre d'imager les échantillons.
%
Cette procédure est incompatible avec l'imagerie par objectif à lentille plongeante.
%
Mais une fois chaque échantillon en place,
il pourrait être possible de les fixer dans leur position en venant placer une goutte de phytagel au dessus de chaque échantillon.
%

La mise en place de tels dispositifs pouvant permettre de monter rapidement un grand nombre d'échantillon,
la simplification du paramétrage de l'imagerie deviendra une étape cruciale pour permettre d'assurer le meilleur rendement de la procédure d'\hti{} développée.

    \subsection{Paramétrage de l'imagerie}

Actuellement, le paramétrage est déjà largement automatisé,
que ce soit par l'emploi du logiciel Matrix Screener pour les microscopes Leica ou bien Jobs pour les microscopes Nikon.
%
Ces logiciels permettent ainsi de fournir une série d'information physiques concernant la plaque à acquérir afin d'automatiser l'acquisition de chaque échantillon.
%
Mais de nombreuses limites existent encore.
%
Je vais donc maintenant décrire deux de ces limites,
tout en proposant à chaque fois une solution.
%
Une première limitation est la détection des échantillons.
%
Il est actuellement obligatoire de demander à un opérateur humain de fournir la position du premier échantillon,
puis de lui faire calibrer la position du premier champ d'acquisition en fonction de l'ensemble des échantillons.
%
En effet, la position des échantillons dans chaque puit n'est pas identique, en fonction de la taille de l'échantillon. 
%
La mise en place d'une procédure de pré-scan en utilisant une loupe fluorescente ou une procédure précédant l'acquisition sur le microscope confocal utilisé pourrait ainsi permettre d'automatiser la détection de la position de chaque échantillon au sein de la plaque fournie.
%
Une fois ces positions connues, il serait alors possible de déterminer automatiquement la position optimale des  champs d'acquisition.
%
Une seconde limitation est la détection du nombre de plan d'acquisitions nécessaires en Z.
%
De manière similaire à la position du premier champ optique,
il est actuellement nécessaire de déterminer les position du premier et du dernier plan d'acquisition permettant l'acquisition entière de l'ensemble des échantillons.
%
Deux éléments rendent cette vérification nécessaire.
%
Le premier provient du montage.
%
La profondeur des encoches faites dans l'agarose n'est pas constante.
%
Ainsi, un échantillon se trouvant plus proche du bord le plus large du puits sera plus bas qu'un échantillon plus proche du bord le plus fin.
%
L'opérateur utilisant la plage de largeurs proposé pour positionner au mieux les échantillons,
les échantillons vont donc présenter plusieurs profondeurs en fonction de la taille de l'échantillon.
%
Un second problème potentiel peut venir d'un mauvais parallélisme entre la platine d'acquisition et l'objectif.
%
Étant donné la faible taille des échantillons imagés et la grande taille des moules, un mauvais positionnement peut entraîner une divergence importante en Z.
%

Cette vérification peut être réalisée lorsque peu d'échantillons sont imagés mais est bien plus fastidieuse lorsqu'il s'agit de vérifier la position en Z de 96 échantillons.
%
Pour résoudre ce problème, sil serait possible de mettre en place un préscan à faible résolution des deux positions limites de chaque échantillon.
%
Durant ce pré-scan, les positions des plans seraient repoussés d'un plan vers l'extérieur si le plan limite étudié contient de l'information.

En conclusion, l'essentiel de l'optimisation de l'imagerie consistera en la reconnaissance par des pré-scans par les microscopes confocaux des échantillons, afin d'éviter les pertes de temps liées à l'acquisition de zones sans échantillons (actuellement programmées pour éviter de rogner l'image d'une larve par une acquisition trop justement programmée)? cette optimisation devrait être facilement réalisable en raison des marqueurs utilisés. Aussi, avec l'augmentation de la taille des échantillons (par exemple des alevins entiers d'un mois) , il sera nécessaire d'augmenter la vitesse d'acquisition, ce qui pourra se faire par de futures générations de microscopes. En effet, à l'heure actuelle, pour de tels échantillons, nous sommes contraints de diminuer la résolution des images jusqu'à quatre microns si plus de douze échantillons sont acquis en une nuit, ce qui pourrait limiter la bonne détection de fibres, vaisseaux petites cellules ou noyaux.

\end{document}
