\providecommand{\main}{../..}

\documentclass[\main/main.tex]{subfiles}

\begin{document}
                    
\section{Vers une augmentation de la quantité de mesures fournies par les segmentations}
%
Dans l'état actuel des algorithmes, la détection d'anomalies est effectuée sur la seule estimation du volume total de chaque région segmentée. Cette stratégie permet la détection de certains défauts dans le développement des tissus, mais ne détecte pas certaines anomalies de formes, qui n'affecteraient pas le volume total de celui-ci, comme la réduction d'une région particulière au profit d'une autre.

Afin d'améliorer les capacités d'analyse permises par ZBI, il sera donc nécessaire de mettre en place de nouvelles mesures.

    \subsection{Les segmentations déjà mises en place pourraient fournir plus d'informations concernant la morphologie des échantillons}

%
En utilisant les algorithmes actuellement développés au cours de cette thèse,
cinq volumes sont segmentés: l'échantillon entier, sa matière blanche, sa matière grise et ses deux cristallins. Avec ces segmentations, il est possible d'extraire cinq volumes supplémentaires.  Le premier est le cerveau entier, obtenu par l'union entre la matière blanche et la matière grise. 
%%
%
En utilisant les segmentations des cristallins et la segmentation de la matière blanche, il est possible d'obtenir la segmentation de chaque oeil dans leur ensemble.
%
Enfin, l'association entre la segmentation de la larve entière et la segmentation du cerveau peut permettre de séparer la tête du corps.
%
Ces dix segmentations peuvent alors être étudiées pour obtenir différentes métriques. Les plus simples sont le volume, la surface et la plus grande longueur de chaque volume. On pourrait aussi développer des mesures de la linéarité de la queue pour identifier d'éventuelles déformations.
%
Concernant d'autres structures, plusieurs métriques supplémentaires semblent intéressantes.
%
La mesure de la sphéricité de chaque segmentation donne une information sur la compacité de chaque volume.
%
De la même manière, il est possible de mesurer l'excentricité de l'ellipsoïde englobant chaque volume, ce qui compare le plus grand et le plus petit demi-axe décrivant l'ellipsoïde, et donne ainsi une information sur la forme de chaque volume.
%
La mesure des harmoniques sphériques de chaque volume pourrait aussi permettre d'améliorer l'analyse de potentiels défauts.

% Ratios
En complément à ces mesures génériques, il est possible de mesurer différents ratios apportant des informations spécifiques. Le ratio entre le volume de matière blanche et le volume total du cerveau donne une estimation des événements de neurogénèse. Le ratio entre volume de l'oeil droit et volume de l'oeil gauche peut informer d'une anomalie dans la symétrie du l'échantillon. Le ratio du volume des yeux sur le volume totale de la larve permet de détecter une microphtalmie.
Le ratio du volume du cristallin gauche sur le volume total de l'oeil gauche informe sur une anomalie de la formation de la rétine.
%
De tels ratios, donnés en exemple, permettraient donc de fournir des informations complémentaires apportant une meilleure description des échantillons.
En utilisant les segmentations obtenues, il est aussi possible d'obtenir des mesures spécifiques à certains organes ou tissus, par exemple la distance entre les deux cristallins, ce qui permet d'avoir une approximation de la largeur du télencéphale et identifier des cyclopies.
Ces mesures ne nécessiteraient pas la mise en place de procédure longues à développer, avec ce qui a été développé au cours de ma thèse..

    \subsection{La mise en place d'un approche utilisant des recalages élastiques permettrait de connaître les déformations locales du cerveau}
    
Grâce à la méthode d'imagerie mise en place au cours de cette thèse,
il est maintenant possible d'imager un grand nombre d'échantillons en peu de temps. A partir de ces nombreux échantillons, il est aussi possible par recalage de créer rapidement un modèle statistique de cerveau de poisson-zèbre pour le stade de développement désiré. La variabilité existante entre les échantillons est gommée pour correspondre à un modèle statistiquement représentatif de \pz.
A TPS, Arnim Jenett développe de telles stratégies.%
Plus le nombre d'échantillons utilisés pour calculer le modèle est grand, plus le modèle statistique se rapprochera de la réalité biologique d'un échantillon prototypique.
%
Ce modèle statistique peut ensuite servir de modèle afin de recaler un échantillon supplémentaire, normal ou anormal. La mesure du champ de déformation ayant permis de recaler cet échantillon sur le modèle statistique est ensuite conservée comme métrique pour des analyses ultérieures.

%
Cette méthode peut de plus être améliorée par l'emploi des atlas. La segmentation d'une liste définie de régions peut être effectuée au sein du modèle statistique. Dans la suite, l'atlas des segmentations manuelles sera nommé atlas statistique.
Une fois ce dernier obtenu, on recale les échantillons que l'on veut étudier sur le modèle statistique. On obtient ainsi une carte des déformations nécessaires pour recaler chaque région de l'échantillon sur le modèle statistique. Ces déformations permettent ensuite d'effectuer la transformation inverse pour transformer l'atlas statistique en segmentation des sous-régions de l'échantillon à traiter.
On obtient pour chaque échantillon une segmentation de chaque sous-région ainsi qu'une carte des déformations nécessaires pour faire correspondre cet échantillon à un modèle statistique.
Il est alors possible de décrire quelles sont les sous-régions les plus fortemment affectées par un polluant ou par une mutation.

En revanche, la mise en place d'une telle stratégie est relativement longue et coûteuse car la création du modèle statistique demande d'imager un grand nombre d'échantillons sauvages. La création manuelle de l'atlas statistique demande l'effort de plusieurs spécialistes pour s'assurer d'avoir un modèle le plus correct possible. Les calculs de recalage dits élastiques sont très lourds, et donc peu adaptés à une stratégie d'\hta. 
Cette stratégie reste néanmoins la plus robuste  pour obtenir un grand nombre de descripteurs,  ce qui peut améliorer l'automatisation de la détection d'anomalies discrètes au sein des échantillons.

\end{document}
