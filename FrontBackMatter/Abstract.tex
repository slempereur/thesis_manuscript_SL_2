% Abstract

%\renewcommand{\abstractname}{Abstract} % Uncomment to change the name of the abstract

\pdfbookmark[1]{Abstract}{Abstract} % Bookmark name visible in a PDF viewer

\begingroup
\let\clearpage\relax
\let\cleardoublepage\relax
\let\cleardoublepage\relax

\chapter*{Résumé}

L'observation d'organismes vivants par microscopie est une activité captivante et esthétique.
%
La portée scientifique des images est indiscutable surtout depuis que les progrès technologiques ont permis l'observation en 3D et au fil du temps de marqueurs fluorescents introduits par transgenèse. Les applications sont innombrables, par exemple pour la compréhension des mécanismes cellulaires et subcellulaires au sein de l'organisme entier. L'espèce modèle poisson zèbre, dont l'utilisation a fortement augmenté dans les laboratoires au cours des trente dernières années, est un modèle de choix pour tirer le meilleur parti des observations microscopiques du fait de sa transparence qui peut-être encore améliorée par des méthodes récentes de clarification de tissus. De plus, sa relativement petite taille permet des observations à grande échelle.
%
L'objectif de ma thèse à été de développer et d'améliorer un ensemble de méthodes, en particulier des algorithmes de traitement d'image, pour mettre au point une plateforme d'observation à haut-contenu et à haut débit du poisson zèbre. Toutes les étapes ont été considérées depuis la préparation d'échantillons, le montage des échantillons pour l'observation microscopique, l'optimisation de la vitesse d'acquisition des microscopes et surtout le développement de méthodes permettant de faciliter la manipulation, l'archivage et l'analyse des images 3D. Au final la méthode permet d'observer une centaine d'échantillons tous alignés de la même façon.
Cet alignement a été obtenu par recalage automatique des images 3D en se basant sur l'axe interoculaire et l'axe principal du cerveau.
De plus, la segmentation de la forme de l'alevin et d'une partie de son cerveau grâce à l'utilisation d'un marqueur lipophile ouvre la possibilité d'analyse quantitative en particulier du volume des différentes structures.
Ces segmentations ont été réalisés par l'utilisation de méthodologies issues de la morphologie mathématique, en particulier la segmentation par ligne de partage des eaux.
La précision et la versatilité des algorithmes a été estimée. La capacité de cette méthode à générer des résultats originaux pour la compréhension de la biologie du poisson zèbre a été testée avec deux projets à portée biomédicale. Le premier concerne l'analyse de mutants possédant des défauts neurodéveloppementaux et le second  l'effet tératogène de traitements anticancéreux. Nous avons de plus montré que la plateforme que j'ai nommée ZeBraInspector permet l'obtention de résultats à la précision considérablement renforcée grâce aux estimations quantitatives en 3D. L'utilisation de cette plateforme a permis une révision des conclusions scientifiques préalablement obtenues ce qui a mené à deux publications. Une des ces publications inclut une description des algorithmes permettant d'effectuer le recalage et la segmentation. Au cours de ma thèse, je me suis aussi aperçu des limites des méthodes de clarification. En effet, la pénétration des tissus induit une atténuation du signal en fonction de la profondeur, ce qui m'a conduit à mettre au point un algorithme de correction du contraste. L'algorithme mis au point est publié dans un troisième article. Un logiciel intuitif pour les utilisateurs a de plus été développé. Je suis convaincu que mon travail de thèse va offrir à la communauté poisson zèbre mais aussi à d'autres communautés travaillant sur d'autres espèces modèles un nouvel outil permettant d'analyser une grande variété de processus biologiques en 3D, à grande échelle et avec une très forte résolution. Des applications supplémentaires sont discutées dans la partie perspectives de ma thèse. Certaines sont déjà en cours de développement au sein de l'unité TEFOR Paris-Saclay où la majeure partie de ma thèse a été effectuée.

\newpage

\chapter*{Abstract}

The observation of living organisms by microscopy is a fascinating and aesthetic activity.
The scientific significance of images is indisputable, especially since technological progress has made it possible to observe in 3D and over time fluorescent markers introduced by transgenesis. The applications are endless, for example for understanding cellular and subcellular mechanisms within the whole organism. The model species zebrafish, the use of which has greatly increased in laboratories over the last thirty years, is a  well-adapted model for microscopic observations because of its transparency which can be further improved by recent methods of tissue clarification. In addition, its relatively small size allows large-scale observations.
The objective of my thesis was to develop and improve a set of methods, in particular image processing algorithms, to develop a high-content and high-throughput platform for zebrafish. All the steps were considered from the preparation of samples, the mounting of the samples for microscopic observation, the optimization of the acquisition speed of the microscopes and especially the development of methods allowing to facilitate the handling, the archiving and the analysis of 3D images. In the end, the method makes it possible to observe around a hundred samples all aligned in the same way.
This alignment was obtained by automatic registration of the 3D images based on the interocular axis and the main axis of the brain.
In addition, the segmentation of the shape of the fry and of a part of its brain through the use of a lipophilic marker opens up the possibility of quantitative analysis, in particular of the volume of the different structures.
These segmentations were carried out using methodologies derived from mathematical morphology, in particular segmentation by watershed.
The precision and versatility of the algorithms was estimated. The capacity of this method to generate original results for understanding the biology of zebrafish has been tested with two projects of biomedical scope. The first concerns the analysis of mutants with neurodevelopmental defects and the second the teratogenic effect of anticancer treatments. We have also shown that the platform that I named ZeBraInspector allows the obtaining of results with considerably improved precision thanks to quantitative 3D estimations. The use of this platform made it possible to revise the scientific conclusions previously obtained, which led to two publications. One of these publications includes a description of the algorithms used to perform registration and segmentation. During my thesis, I also noticed the limits of the methods of clarification. Indeed, tissue penetration induces an attenuation of the signal as a function of depth, which led me to develop a contrast correction algorithm. The algorithm developed is published in a third article. In addition, an user-friendly software has been developed. I am convinced that my thesis work will offer the zebrafish community but also to other communities working on other model species a new tool allowing to analyze a wide variety of biological processes in 3D, on a large scale and with very high resolution. Additional applications are discussed in the “perspective” part of my thesis. Some are already under development within the TEFOR Paris-Saclay unit where most of my thesis was carried out.


\endgroup			

\vfill